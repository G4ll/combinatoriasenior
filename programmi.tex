\documentclass[a4paper,10pt]{article}

\usepackage[utf8]{inputenc}
\usepackage[italian]{babel}

\usepackage{amsmath}


%opening
\title{Programmi di combinatoria}
\author{}

\begin{document}

\maketitle

\section{Basic}
\subsection{Prerequisiti}
Le nozioni di base sui conteggi ossia cose come fattoriale, definizione di binomiale e conteggi standard tipo combinazioni, permutazioni, disposizioni. Ecco alcuni esercizi che vorrei dare per scontato con i prerequisiti.
\paragraph{Esercizio 1.} In una gara podistica prendono parte 10 atleti. Quanti sono i possibili podi a fine gara?
\paragraph{Esercizio 2.} Quante sono le possibili cinquine di numeri estratti al lotto?
\paragraph{Esercizio 3.} Quanti sono i possibili anagrammi (compresi quelli senza senso) della parola aiuole?
\paragraph{Esercizio 4.} In quanti modi è possibile scegliere un gruppetto di persone
(eventualmente vuoto) tra 20 studenti?


\subsection{C1: Conteggi e double counting}
In qualche punto tra qui e C2 andrebbe inserita una minima trattazione della probabilità (forse, qui definizione ed esempi, e in C2 qualche fenomeno minimamente più complicato).

\subsubsection{Conteggi elementari}
Disposizioni, combinazioni, permutazioni ovvero anagrammi e targhe. Coefficienti binomiali e loro proprietà, sia algebriche che combinatoriche. Vari esempi di come ricondurre un problema ad un altro.

\subsubsection{Tecniche di conteggio}
Double counting, inclusione-esclusione, ”classi di equivalenza”.


\subsection{C2}
\subsubsection{Grafi}
Definizione, esempi, conteggi sui vertici, archi, valenze. Double counting classici. Grafi euleriani. Campionati. Formula di Eulero.

\subsubsection{Colorazioni ed invarianti}
Esempi di utilizzo delle colorazioni, concetto di invariante, esempi e controesempi.


\section{Medium}
\subsection{Prerequisiti}
\subsection{C1: Tecniche di non esistenza}
\subsection{C2: Tecniche di esistenza non costruttiva}
\subsection{C3: Tecniche di esistenza costruttiva}


\section{Advanced}
\subsection{Prerequisiti}
\subsection{Argomenti}

\end{document}
